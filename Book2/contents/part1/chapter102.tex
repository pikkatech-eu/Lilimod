\chapter {Laplace Transform}
	\section {Definitions}
		\lt is a major tool for the analysis of behavior of \ldsys. Here we render a few general definition of this instrument and point to those of its properties that find application for those purposes. We limit ourselves to a mere enumeration of the facts, with a minimum of evaluation or proofs. Readers interested in strict mathematical results are referred to a host of available literature (e.g. \cite{davies_integral_2002}).
	
	\lt is defined for functions of time $f(t)$ with the following properties:
	\begin{itemize}\itemsep-4pt
		\item $f(t) = 0, t < 0;$
		\item for $t \rightarrow \infty $, $f(t)$ either decreases or, if it increases, then no faster than some exponential, $f(t) \leq e^{\alpha t}, \alpha > 0.$
	\end{itemize}
	Functions of time occurring in the study of \ldys normally fulfill both of these conditions.
	
	If the above conditions are fulfilled, the Laplace image of $f(t)$ is defined as
	
	\begin{equation}\label{eq:lablace basic}
		F(p) = \int \limits_0^\infty e^{-pt} f(t) dt.
	\end{equation}
	
	The function $F(t)$ is then called the Laplace original of $F(p).$
	
	Often the fact that $F(p)$ is the Laplace image of $f(t)$ is represented in operator notation:
	\begin{equation}
		F(p) = \lop{f(t)}.
	\end{equation}
	
	The other way round, if we want to state that $f(t)$ is the Laplace original of $F(p)$ we can express it with the following operator notation:
	\begin{equation}
		f(t) = \ilop{F(p)}.
	\end{equation}
	
	\section{Some properties of \lt}
		\subsection{Linearity\\}
		\lt is linear, that is the principle of superposition fulfills (obviously, due to the linearity of the operation of integration):
		\begin{equation}\label{eq:laplace lineariry}
			\begin{aligned}
				\lop{f_1(t) + f_2(t)}  &= \lop{f_1(t)} + \lop{f_2(t)}, \\
				\lop{k \cdot f(t)} &= k \cdot \lop{f(t)}, k = const.
			\end{aligned}
		\end{equation}
		
		\subsection{\lt of derivative\\}
		If the \lt for a function is known, \lt for its derivative can be found in a simple expression. Indeed,
		\[ 
		\lop{f'(t)} = \int \limits_0^\infty e^{-pt} f'(t) dt.
		\] 
		Applying the rule of partial integration ($e^{-pt} = u, ~ f't) dt = dv$), we obtain that 
		\[ \lop{f'(t)} = e^{-pt} f(t) \rvert_0^\infty + p \int \limits_0^\infty e^{-pt} f(t) dt, \] in other words,
		
		\begin{equation}
			\lop{f'(t)} = p \lop{f(t)} - f(0).
		\end{equation}
		
		If, additionally, $f(0) = 0$, we have
		\begin{equation}\label{eq: laplace of derivative}
			\lop{f'(t)} = p \cdot \lop{f(t)}.
		\end{equation}
		
		\subsection{\lt of higher derivatives\\}
		Analog to how the expression for the \lt of the first derivative was obtained, the same technique can be applied to obtain expressions for the \lt of higher derivatives. For example,
		\begin{equation}\label{eq: lt of higher derivatives}
			\begin{aligned}
				\lop{f''(t)} &= p^2 \lop{f(t)} - f(0) p - f'(0), \\
				\lop{f'''(t)} &= p^3 \lop{f(t)} - f(0) p^2 - f'(0) p - f''(0), \\
				etc.&~
			\end{aligned}
		\end{equation}
		
		\subsection{\lt of integral\\}
		Let $g(t) = \int \limits_0^t f(\tau) d \tau$ be the integral function of $f(t.)$ The \lt of $g(t)$ can then be expressed in terms of $\lop{f(t)}.$ Indeed, applying the rule of partial integration to the expression
		
		\[
		\int \limits_0^\infty e^{-pt} g(t) dt
		\]
		
		we let $e^{-pt} dt = dv,~ g(t) = u.$ Then $v = -\frac{1}{p} \cdot e^{-pt}, ~ du = \frac{dg}{dt} \cdot dt = f(t) dt,$ and
		
		\[ 
		\lop{g(t)} = -\frac{1}{p} \cdot g(t) \cdot e^{-pt} \rvert_0^\infty + \frac{1}{p} \cdot \int \limits_0^\infty e^{-pt} f(t) dt,
		\]
		
		or
		
		\begin{equation}\label{eq: lt of integral}
			\lop{\int \limits_0^t f(\tau) d \tau} = \frac{1}{p} \cdot \lop{f(t)}.
		\end{equation}
		
		\subsection{\lt of repeated integral\\}
		Analog to how the expression for the \lt of the integral of a function was obtained, the same technique can be applied to obtain expressions for the \lt of repeated integrals. For example, if $g(t) = \int \limits_0^t f(\tau) d \tau,$
		
		\begin{equation} 
			\lop{\int \limits_0^t g(\sigma) d \sigma} = \frac{1}{p^2} \cdot \lop{f(t).}
		\end{equation}
		
		\subsection{Limit values\\}
		The values ​​of the original $f(t)$ at $t = 0$ and $t \rightarrow \inf$ an be determined using the image $F(p) = \lop{f(t)}$ as follows:
		\begin{equation}
			\begin{aligned}
				f(0) &= \lim_{p \to \infty} p F(p), \\
				f(\infty) &= \lim_{p \to 0} p F(p). \\
			\end{aligned}
		\end{equation}
		
		\subsection{Time shift\\}
		If $F(p = \lop{f(t)}),$ then the \lt of the same function shifted by the value $a$
		along the time axis, $f(t - a), $ is given by the following expression:
		\begin{equation}\label{eq: laplace shift}
			\lop{f(t+a)} = e^{-pa} \cdot \lop{f(t).}
		\end{equation}
		To obtain this, it is sufficient to perform a simple substitution of variables in  equation [\ref{eq:lablace basic}]. (Notice at that the values of the shifted function are zero if $t \in [0, a]$!)
		
	\section{Images of some basic functions} \label{sec:Laplace image examples}
		\subsection{Heaviside's unitary step}
		Heaviside's unitary step function is defined as 
		
		\begin{equation} \label{eq:heaviside}
			H(t) = \left\{\begin{array}{ll} 		% 'll' aligns two columns to the left
				0, & \text{if } t < 0. \\ 			% Use \text{} for text, & to separate
				1, & \text{if } t \ge 0.
			\end{array}
			\right.
		\end{equation}
		
		Its \lt is 
		\begin{equation}\label{eq:laplace heaviside}
			\lop{H(t)} = \frac{1}{p}.
		\end{equation}
		
		\subsection{Linear function}
		For the linear function
		\[
		f(t) = \left\{\begin{array}{ll}
			0, & \text{if } t < 0. \\
			\alpha t, & \text{if } t \ge 0 ~(\alpha = const)
		\end{array}
		\right.
		\]
		the \lt is 
		\begin{equation}\label{eq: lt linear}
			\lop{\alpha t} = \frac{\alpha}{p^2}.
		\end{equation}
		
		\subsection{Exponential function}
		For the exponential function $f(t) = e^{\alpha t}$ the \lt is
		\begin{equation}\label{eq: lt exponent}
			\lop{e^{\alpha t}} = \frac{1}{p - \alpha}.
		\end{equation}
		
		\subsection{Dirac's delta function}
	Dirac's delta function is a so called generalized function that can only be defined as a limit of a series of "normal" functions. Because of the special importance of this function in many applications (it describes, among other properties, locations of infinitely small sources and sinks), it seems to be appropriate to explain it to some more extent.
	
	\begin{figure}[H]
		\centering
		\includegraphics[width=0.5\linewidth]{contents/part1/figures/diracs_delta}
		\caption{To the definition of Diracs delta function}
		\label{fig:diracs-delta}
	\end{figure}
	
	There are many possibilities to define Dirac's delta function. The simplest of them is to consider the sequence of step functions like those shown in Figure  \ref{fig:diracs-delta}:
	
	\begin{equation}
		h_\varepsilon(t) = \left\{\begin{array}{ll}
			\frac{1}{\varepsilon}, \text{if } 0 \leq t <= \varepsilon, \\
			0 \mathrm{otherwise}
		\end{array}
		\right.
	\end{equation}
	
	With any finite value of $\varepsilon$ the integral $\int \limits_0^\infty h_\varepsilon(t) dt, $ or the area below the $h_\varepsilon$ line preserves the value of 1.
	
	Dirac's delta function $ \delta(t)$ is regarded as the generalized limit of this sequence of $h_\varepsilon(t)$ with $t \rightarrow 0.$
	
	According to the equations [\ref{eq:laplace heaviside}], [\ref{eq:laplace lineariry}], [\ref{eq: laplace shift}] the \lt of $h_\varepsilon(t)$ at any finite value of $\varepsilon$ is
	\[ 
	\lop{h_\varepsilon(t)} = \frac{1-e^{-p\varepsilon}}{p\varepsilon}.
	\]
	
	With $\varepsilon \rightarrow 0 $ the numerator of the latter equation approaches $p\cdot \varepsilon + o(p \varepsilon),$ therefore $\lim_{\varepsilon \to 0} \lop{h_\varepsilon(t)} = 1,$ in other words,
	\begin{equation} \label{eq:lt delta}
		\lop{\delta(t)} = 1.
	\end{equation}
	
	\section {Solution of \odes}
	Using the technique of \lt, we now can obtain the solution of any linear \ode with constant coefficients and an arbitrary right-part function. 
	
	Consider again the linear \ode \eqref{eq: linear ode general} 
	Applying \lt to both parts of (\ref{eq: general linear ode}), we obtain:
	\[
	\left( \sum \limits_{k=0}^{n} a_k p^k \right) \cdot \hat Y(p) = \hat F(p) + P_0(p),
	\]
	
	where $\hat{F}(p) = \lop{f(t)}, $ $P_0(p)$ a polynomial of degree $n-1$ with coefficients being linear combinations of the initial values of the function $y(t)$ and its derivatives.
	
	In the special case when $P_0(p) = 0$,
	
	\begin{equation}
		\hat Y(p) = U(p) \cdot \hat F(p),
	\end{equation}
	
	where
	\begin{equation}\label{eq: transfer function}
		U(p) = \frac{1}{\sum \limits_{k=0}^{n} a_k p^k}.
	\end{equation}
	
	In (\ref*{eq: transfer function}) $U(p)$ is called the transfer function \index{transfer function} of the \ldys. 

	\subsection{Example. Solution of \ode using \lt}
	Let us again consider \ode \eqref{eq: simple linear ode example 1} with initial conditions  $y(0) = 0, ~y'(0) = 1$. Let us set $\hat{Y} = \hat{Y}(p) = \lop{y(t)}$, then, according to \eqref{eq: lt of higher derivatives} $p \hat{Y} - y(0) = \lop{y'(t)}$ and $p ^2 \hat{Y}- p y(0) - y'(0) = \lop{y''(t)}$. Since $y(0) = 0$ and $y'(0) = 1$ in the example, we have:
	\begin{equation}
		\begin{aligned}
			\lop{y(t)} &= \hat{Y(p)},\\
			\lop{y'(t)} &= p \hat{Y(p)}, \\
			\lop{y''(t)} &= p^2 \hat{Y(p)} - 1.
		\end{aligned}
	\end{equation}
	The Laplace image ot the right part function $k t$ is, according to the table \ref{tab: table of laplace transforms} of the Appendix (\#3) $\ds \lop{ k t} = \frac{k}{p ^2}$. In \lt terms, we converted the \de \eqref{eq: simple linear ode example 1} to the following algebraic equation:
	\begin{equation}\label{eq:example 2 in laplace images}
		p^2 \hat{Y} - 1 + p \hat{Y} - 2 \hat{Y} = \frac{k}{p ^2},
	\end{equation}
	or, after elementary transformations,
	\begin{equation}
		\hat{Y} = \frac{k}{p ^2} \cdot \frac{1}{p^2 + p - 2} - \frac{1}{p^2 + p - 2}.
	\end{equation}
	
	Expression $\ds \frac{1}{p^2 + p - 2}$ can be transformed, using the method of undefined coefficients, to the following form:
	\begin{equation}
		\frac{1}{p^2 + p - 2} = \frac{1}{3} \cdot \frac{1}{p - 1} - \frac{1}{3} \cdot \frac{1}{p + 2}.
	\end{equation}
	Now, to invert the expression \eqref{eq:example 2 in laplace images} (and thus to obtain the solution of the \de as a function of time) we need to perform the following four elementary conversions:
	\[ 
	\begin{aligned}
		\frac{1}{3} \frac{k}{p^2} \cdot \frac{1}{p - 1} &~, \\
		\frac{1}{3} \frac{k}{p^2} \cdot \frac{1}{p + 2} &~,\\
		\frac{1}{3}  \cdot  \frac{1}{p - 1} &~,\\
		\frac{1}{3}  \cdot  \frac{1}{p + 2} &\\
	\end{aligned}
	\]
	and combine them into a sum taking into consideration the signs before each of these terms. The originals of the last two terms are simple exponents (table \ref{tab: table of laplace transforms} of the Appendix (\#8)): $\ds \E^t$ and $\ds \E^{-2t}$ respectively; the originals for the first two terms are also found in table \ref{tab: table of laplace transforms}.
	
	After performing all necessary algebraic transformation, we come to the solution we found for this example in \ref{sec:Non-homogeneous equation}:
	\[ 
		y(t) = \frac{k + 1}{3} \E^t -\frac{k + 4}{12} \E^{-2t} - \frac{k}{4} (2t + 1).
	\]
	\section {Inversion}
	In the section above description of a \ldys in terms of a differential equation has been reduced to a formally algebraic equation with respect to the Laplacian variable $p$, and this with an arbitrary right-part function. For all practical purposes, however, it is not enough: what we want to know is the \tmp behavior of the solution. In other words, we need convert the Laplace image of a function to the \tmp dependency itself, i.d. to invert the \lt.
	
	In section \ref{sec:Laplace image examples} we have derived a few elementary correspondences between \tmp functions and their Laplace images: 
	(\ref{eq:laplace heaviside}) for the Heaviside function, (\ref{eq: lt linear}) for the linear function, (\ref{eq: lt exponent}) for the exponential function, and (\ref{eq:lt delta}) for Dirac's delta function. 
	
	Table \ref{tab:laplace} in Appendix contains more cases arises in concrete problems in \gf and similar contexts. There exist, besides, numerous collections of comprehensive inversion tables. Classical sources are, for example, \cite{Bateman_1954}, \cite{oberhettinger_tables_1973}, \cite{Abramowitz1964}, and many others.
	
	\section {Tables}
	\section {Numerical inversion}
	\subsection {Stehfest}
	\subsection {den Iseger}