\chapter {Laplace Transform}
	\section {Definitions}
		\lt is a major tool for the analysis of behavior of \ldsys. Here we render a few general definition of this instrument and point to those of its properties that find application for those purposes. We limit ourselves to a mere enumeration of the facts, with a minimum of evaluation or proofs. Readers interested in strict mathematical results are referred to a host of available literature (e.g. \cite{davies_integral_2002}).
	
	\lt is defined for functions of time $f(t)$ with the following properties:
	\begin{itemize}\itemsep-4pt
		\item $f(t) = 0, t < 0;$
		\item for $t \rightarrow \infty $, $f(t)$ either decreases or, if it increases, then no faster than some exponential, $f(t) \leq e^{\alpha t}, \alpha > 0.$
	\end{itemize}
	Functions of time occurring in the study of \ldys normally fulfill both of these conditions.
	
	If the above conditions are fulfilled, the Laplace image of $f(t)$ is defined as
	
	\begin{equation}\label{eq:lablace basic}
		F(p) = \int \limits_0^\infty e^{-pt} f(t) dt.
	\end{equation}
	
	The function $F(t)$ is then called the Laplace original of $F(p).$
	
	Often the fact that $F(p)$ is the Laplace image of $f(t)$ is represented in operator notation:
	\begin{equation}
		F(p) = \lop{f(t)}.
	\end{equation}
	
	The other way round, if we want to state that $f(t)$ is the Laplace original of $F(p)$ we can express it with the following operator notation:
	\begin{equation}
		f(t) = \ilop{F(p)}.
	\end{equation}
	
	\section{Some properties of \lt}
		\subsection{Linearity\\}
		\lt is linear, that is the principle of superposition fulfills (obviously, due to the linearity of the operation of integration):
		\begin{equation}\label{eq:laplace lineariry}
			\begin{aligned}
				\lop{f_1(t) + f_2(t)}  &= \lop{f_1(t)} + \lop{f_2(t)}, \\
				\lop{k \cdot f(t)} &= k \cdot \lop{f(t)}, k = const.
			\end{aligned}
		\end{equation}
		
		\subsection{\lt of derivative\\}
		If the \lt for a function is known, \lt for its derivative can be found in a simple expression. Indeed,
		\[ 
		\lop{f'(t)} = \int \limits_0^\infty e^{-pt} f'(t) dt.
		\] 
		Applying the rule of partial integration ($e^{-pt} = u, ~ f't) dt = dv$), we obtain that 
		\[ \lop{f'(t)} = e^{-pt} f(t) \rvert_0^\infty + p \int \limits_0^\infty e^{-pt} f(t) dt, \] in other words,
		
		\begin{equation}
			\lop{f'(t)} = p \lop{f(t)} - f(0).
		\end{equation}
		
		If, additionally, $f(0) = 0$, we have
		\begin{equation}\label{eq: laplace of derivative}
			\lop{f'(t)} = p \cdot \lop{f(t)}.
		\end{equation}
		
		\subsection{\lt of higher derivatives\\}
		Analog to how the expression for the \lt of the first derivative was obtained, the same technique can be applied to obtain expressions for the \lt of higher derivatives. For example,
		\begin{equation}\label{eq: lt of higher derivatives}
			\begin{aligned}
				\lop{f''(t)} &= p^2 \lop{f(t)} - f(0) p - f'(0), \\
				\lop{f'''(t)} &= p^3 \lop{f(t)} - f(0) p^2 - f'(0) p - f''(0), \\
				etc.&~
			\end{aligned}
		\end{equation}
		
		\subsection{\lt of integral\\}
		Let $g(t) = \int \limits_0^t f(\tau) d \tau$ be the integral function of $f(t.)$ The \lt of $g(t)$ can then be expressed in terms of $\lop{f(t)}.$ Indeed, applying the rule of partial integration to the expression
		
		\[
		\int \limits_0^\infty e^{-pt} g(t) dt
		\]
		
		we let $e^{-pt} dt = dv,~ g(t) = u.$ Then $v = -\frac{1}{p} \cdot e^{-pt}, ~ du = \frac{dg}{dt} \cdot dt = f(t) dt,$ and
		
		\[ 
		\lop{g(t)} = -\frac{1}{p} \cdot g(t) \cdot e^{-pt} \rvert_0^\infty + \frac{1}{p} \cdot \int \limits_0^\infty e^{-pt} f(t) dt,
		\]
		
		or
		
		\begin{equation}\label{eq: lt of integral}
			\lop{\int \limits_0^t f(\tau) d \tau} = \frac{1}{p} \cdot \lop{f(t)}.
		\end{equation}
		
		\subsection{\lt of repeated integral\\}
		Analog to how the expression for the \lt of the integral of a function was obtained, the same technique can be applied to obtain expressions for the \lt of repeated integrals. For example, if $g(t) = \int \limits_0^t f(\tau) d \tau,$
		
		\begin{equation} 
			\lop{\int \limits_0^t g(\sigma) d \sigma} = \frac{1}{p^2} \cdot \lop{f(t).}
		\end{equation}
		
		\subsection{Limit values\\}
		The values ​​of the original $f(t)$ at $t = 0$ and $t \rightarrow \inf$ an be determined using the image $F(p) = \lop{f(t)}$ as follows:
		\begin{equation}
			\begin{aligned}
				f(0) &= \lim_{p \to \infty} p F(p), \\
				f(\infty) &= \lim_{p \to 0} p F(p). \\
			\end{aligned}
		\end{equation}
		
		\subsection{Time shift\\}
		If $F(p = \lop{f(t)}),$ then the \lt of the same function shifted by the value $a$
		along the time axis, $f(t - a), $ is given by the following expression:
		\begin{equation}\label{eq: laplace shift}
			\lop{f(t+a)} = e^{-pa} \cdot \lop{f(t).}
		\end{equation}
		To obtain this, it is sufficient to perform a simple substitution of variables in  equation [\ref{eq:lablace basic}]. (Notice at that the values of the shifted function are zero if $t \in [0, a]$!)
		
	\section{Images of some basic functions} \label{sec:Laplace image examples}
		\subsection{Heaviside's unitary step}
		Heaviside's unitary step function is defined as 
		
		\begin{equation} \label{eq:heaviside}
			H(t) = \left\{\begin{array}{ll} 		% 'll' aligns two columns to the left
				0, & \text{if } t < 0. \\ 			% Use \text{} for text, & to separate
				1, & \text{if } t \ge 0.
			\end{array}
			\right.
		\end{equation}
		
		Its \lt is 
		\begin{equation}\label{eq:laplace heaviside}
			\lop{H(t)} = \frac{1}{p}.
		\end{equation}
		
		\subsection{Linear function}
		For the linear function
		\[
		f(t) = \left\{\begin{array}{ll}
			0, & \text{if } t < 0. \\
			\alpha t, & \text{if } t \ge 0 ~(\alpha = const)
		\end{array}
		\right.
		\]
		the \lt is 
		\begin{equation}\label{eq: lt linear}
			\lop{\alpha t} = \frac{\alpha}{p^2}.
		\end{equation}
		
		\subsection{Exponential function}
		For the exponential function $f(t) = e^{\alpha t}$ the \lt is
		\begin{equation}\label{eq: lt exponent}
			\lop{e^{\alpha t}} = \frac{1}{p - \alpha}.
		\end{equation}
		
		\subsection{Dirac's delta function}
	Dirac's delta function is a so called generalized function that can only be defined as a limit of a series of "normal" functions. Because of the special importance of this function in many applications (it describes, among other properties, locations of infinitely small sources and sinks), it seems to be appropriate to explain it to some more extent.
	
	\begin{figure}[H]
		\centering
		\includegraphics[width=0.5\linewidth]{contents/part1/figures/diracs_delta}
		\caption{To the definition of Diracs delta function}
		\label{fig:diracs-delta}
	\end{figure}
	
	There are many possibilities to define Dirac's delta function. The simplest of them is to consider the sequence of step functions like those shown in Figure  \ref{fig:diracs-delta}:
	
	\begin{equation}
		h_\varepsilon(t) = \left\{\begin{array}{ll}
			\frac{1}{\varepsilon}, \text{if } 0 \leq t <= \varepsilon, \\
			0 \mathrm{otherwise}
		\end{array}
		\right.
	\end{equation}
	
	With any finite value of $\varepsilon$ the integral $\int \limits_0^\infty h_\varepsilon(t) dt, $ or the area below the $h_\varepsilon$ line preserves the value of 1.
	
	Dirac's delta function $ \delta(t)$ is regarded as the generalized limit of this sequence of $h_\varepsilon(t)$ with $t \rightarrow 0.$
	
	According to the equations [\ref{eq:laplace heaviside}], [\ref{eq:laplace lineariry}], [\ref{eq: laplace shift}] the \lt of $h_\varepsilon(t)$ at any finite value of $\varepsilon$ is
	\[ 
	\lop{h_\varepsilon(t)} = \frac{1-e^{-p\varepsilon}}{p\varepsilon}.
	\]
	
	With $\varepsilon \rightarrow 0 $ the numerator of the latter equation approaches $p\cdot \varepsilon + o(p \varepsilon),$ therefore $\lim_{\varepsilon \to 0} \lop{h_\varepsilon(t)} = 1,$ in other words,
	\begin{equation} \label{eq:lt delta}
		\lop{\delta(t)} = 1.
	\end{equation}
	
	\section {Solution of \odes}
	Using the technique of \lt, we now can obtain the solution of any linear \ode with constant coefficients and an arbitrary right-part function. 
	
	Consider again the linear \ode \eqref{eq: linear ode general} 
	Applying \lt to both parts of (\ref{eq: general linear ode}), we obtain:
	\[
	\left( \sum \limits_{k=0}^{n} a_k p^k \right) \cdot \hat Y(p) = \hat F(p) + P_0(p),
	\]
	
	where $\hat{F}(p) = \lop{f(t)}, $ $P_0(p)$ a polynomial of degree $n-1$ with coefficients being linear combinations of the initial values of the function $y(t)$ and its derivatives.
	
	In the special case when $P_0(p) = 0$,
	
	\begin{equation}
		\hat Y(p) = U(p) \cdot \hat F(p),
	\end{equation}
	
	where
	\begin{equation}\label{eq: transfer function}
		U(p) = \frac{1}{\sum \limits_{k=0}^{n} a_k p^k}.
	\end{equation}
	
	In (\ref*{eq: transfer function}) $U(p)$ is called the transfer function \index{transfer function} of the \ldys. 

	\subsection{Example. Solution of \ode using \lt}
	Let us again consider \ode \eqref{eq: simple linear ode example 1} with initial conditions  $y(0) = 0, ~y'(0) = 1$. Let us set $\hat{Y} = \hat{Y}(p) = \lop{y(t)}$, then, according to \eqref{eq: lt of higher derivatives} $p \hat{Y} - y(0) = \lop{y'(t)}$ and $p ^2 \hat{Y}- p y(0) - y'(0) = \lop{y''(t)}$. Since $y(0) = 0$ and $y'(0) = 1$ in the example, we have:
	\begin{equation}
		\begin{aligned}
			\lop{y(t)} &= \hat{Y(p)},\\
			\lop{y'(t)} &= p \hat{Y(p)}, \\
			\lop{y''(t)} &= p^2 \hat{Y(p)} - 1.
		\end{aligned}
	\end{equation}
	The Laplace image ot the right part function $k t$ is, according to the table \ref{tab: table of laplace transforms} of the Appendix (\#3) $\ds \lop{ k t} = \frac{k}{p ^2}$. In \lt terms, we converted the \de \eqref{eq: simple linear ode example 1} to the following algebraic equation:
	\begin{equation}\label{eq:example 2 in laplace images}
		p^2 \hat{Y} - 1 + p \hat{Y} - 2 \hat{Y} = \frac{k}{p ^2},
	\end{equation}
	or, after elementary transformations,
	\begin{equation}
		\hat{Y} = \frac{k}{p ^2} \cdot \frac{1}{p^2 + p - 2} - \frac{1}{p^2 + p - 2}.
	\end{equation}
	
	Expression $\ds \frac{1}{p^2 + p - 2}$ can be transformed, using the method of undefined coefficients, to the following form:
	\begin{equation}
		\frac{1}{p^2 + p - 2} = \frac{1}{3} \cdot \frac{1}{p - 1} - \frac{1}{3} \cdot \frac{1}{p + 2}.
	\end{equation}
	Now, to invert the expression \eqref{eq:example 2 in laplace images} (and thus to obtain the solution of the \de as a function of time) we need to perform the following four elementary conversions:
	\[ 
	\begin{aligned}
		\frac{1}{3} \frac{k}{p^2} \cdot \frac{1}{p - 1} &~, \\
		\frac{1}{3} \frac{k}{p^2} \cdot \frac{1}{p + 2} &~,\\
		\frac{1}{3}  \cdot  \frac{1}{p - 1} &~,\\
		\frac{1}{3}  \cdot  \frac{1}{p + 2} &\\
	\end{aligned}
	\]
	and combine them into a sum taking into consideration the signs before each of these terms. The originals of the last two terms are simple exponents (table \ref{tab: table of laplace transforms} of the Appendix (\#8)): $\ds \E^t$ and $\ds \E^{-2t}$ respectively; the originals for the first two terms are also found in table \ref{tab: table of laplace transforms}.
	
	After performing all necessary algebraic transformation, we come to the solution we found for this example in \ref{sec:Non-homogeneous equation}:
	\[ 
		y(t) = \frac{k + 1}{3} \E^t -\frac{k + 4}{12} \E^{-2t} - \frac{k}{4} (2t + 1).
	\]
	\section {Inversion}
	
	In the section above description of a \ldys in terms of a differential equation has been reduced to a formally algebraic equation with respect to the Laplacian variable $p$, and this with an arbitrary right-part function. For all practical purposes, however, it is not enough: what we want to know is the \tmp behavior of the solution. In other words, we need convert the Laplace image of a function to the \tmp dependency itself, i.d. to invert the \lt.
	
	In section \ref{sec:Laplace image examples} we have derived a few elementary correspondences between \tmp functions and their Laplace images: 
	(\ref{eq:laplace heaviside}) for the Heaviside function, (\ref{eq: lt linear}) for the linear function, (\ref{eq: lt exponent}) for the exponential function, and (\ref{eq:lt delta}) for Dirac's delta function. 
	
	Table \ref{tab: table of laplace transforms} in Appendix contains more cases arises in concrete problems in \gf and similar contexts. There exist, besides, numerous collections of comprehensive inversion tables. Classical sources are, for example, \cite{Bateman_1954}, \cite{oberhettinger_tables_1973}, \cite{Abramowitz1964}, and many others.
	
	Very often, however, even the most complete tables of \lti do not give an unequivocal answer for a problem. In these cases, numerical inversion of \lt must be used.
	
	%\section {Tables}
	\section {Numerical inversion}
	Despite the fact that the collection of \lt and its inversion is very ample, it is not always possible to find one in existing sources, especially by complicated or exotic \tmp dependencies of external forces. Fortunately, there also exists methods to find the numerical values of a Laplace original if the image is known. This class of methods is known as the numerical inversion of \lt.
	
	A great number of methods for the numerical inversion of \lt has been developed in the last decades. The interested reader can find an overview of existing methods, among other publications, in \cite{davies_martin_1979}, \cite{cohen_numerical_2007}, \cite{kuhlman_2013}, \cite{piessens_1975}.
	
	According to the author's opinion, however, two methods stand out with respect of practical applications: the method of Harald Stehfest \cite{stehfest_1970} and the method of Peter den Iseger \cite{iseger_2006}. 
	%In the following sections we will dwell upon these two methods, and on one further historical method of Lev Gokhberg \cite{gokhberg_1982} included rather for private considerations.
	
	\subsection {Method of Stehfest}
	
	The method of Stehfest uses only the values of \lti om the real axis. This makes it a practical tool, since calculation of the image function with complex arguments causes sometimes greater difficulties. The method is based on an integration formula given by D.P. Gaver \cite{gaver_1966}:
	
	\begin{equation}\label{eq:stehfest}
		f(t) \approx \frac{\log 2}{t} \cdot \sum \limits_{i=1}^N V_i F\left( \frac{\log 2}{t} \cdot i \right),
	\end{equation}
	where
	
	\begin{description}[labelwidth=4em,leftmargin=!,itemsep=-4pt]
		\item[$f(t)$] the approximate value of the original at time value $t$,
		\item[$F(p)$] the Laplace image function,
		\item[$V_i$] the coefficients of integration in Gaver's equation,
		\item[$N$] the order of the integration formula.
	\end{description}
	
	The expression for the coefficients $V_i$ is as follows:
	\begin{equation}\label{eq: gaver coefficients}
		V_i = (-1)^{i + N/2} \cdot \sum \limits_{k = \left[ \frac{i + 1}{2} \right]}^{\min(i, N/2)} \frac{k^{N/2 + 1} (2k)!}{(N/2 - k)! k! (k-1)! (i-k)! (2k-i)!}
	\end{equation}
	
	Since the coefficients $V_i$ only depend on the value of $N$, so they only need to be calculated once. The value of $N$ must be even.
	
	The values of Stehfest's coefficients for a few values of $N$ are given in Table \ref*{tab:Stehfest coefficients}.
	
	\begin{table}[H]\label{tab:Stehfest coefficients}
		\centering
		\caption{Table of coefficients in Stehfest's inversion method fo a few values of $N$}
		\begin{tabular}{|r|r|r|r|r|r|r|}
			\hline
			i & N=2  & N=4     & N=6     & N=8            & N=10            & N=12\\
			\hline
			1 &  2.0 &   -2.0  &     1.0 &      -0.333333 &        0.083333 &  -1.666667e-02 \\
			2 & -2.0 &   26.0  &   -49.0 &      48.333333 &      -32.083333 &   1.601667e+01 \\
			3 &      &   -48.0 &   366.0 &    -906.000000 &     1279.000000 &  -1.247000e+03 \\
			4 &      &    24.0 &  -858.0 &    5464.666667 &   -15623.666667 &   2.755433e+04 \\
			5 &      &         &   810.0 &  -14376.666667 &    84244.166667 &  -2.632808e+05 \\
			6 &      &         &  -270.0 &   18730.000000 &  -236957.500000 &   1.324139e+06 \\
			7 &      &         &         &  -11946.666667 &   375911.666667 &  -3.891706e+06 \\
			8 &      &         &         &    2986.666667 &  -340071.666667 &   7.053286e+06 \\
			9 &      &         &         &                &   164062.500000 &  -8.005336e+06 \\
			10 &     &         &         &                &   -32812.500000 &   5.552830e+06 \\
			11 &     &         &         &                &                 &  -2.155507e+06 \\
			12 &     &         &         &                &                 &   3.592512e+05 \\
			\hline
		\end{tabular}			
	\end{table}
	
	
	
	A Python implementation of Stehfest's algorithm is contained in the \texttt{lilimod} library (\texttt{lilimaths/stehfest.py}).
	
	\subsubsection{Examples of application of Stehfest's method}
	
	\paragraph{Exponential Decay\\~}
	Laplace image: $F(p) = \frac{1}{1 + p}.$ Exact Laplace original: $f(t) = \E^{-t}.$ \\
	
	The results of numerical inversion are shown in Figure \ref{fig:stehfestexponentialdecay}. In this case, even with a relatively lower value of N=6 the results are already very accurate (maximum absolute deviation 0.0068 with $N=6$ at $t = 3.11.$)
	
	\begin{figure}[H]
		\centering
		\includegraphics[width=0.85\linewidth]{contents/part1/figures/stehfest_exponential_decay}
		\caption{Results of numerical Laplace inversion by Stehfest's method for exponential decay.}
		\label{fig:stehfestexponentialdecay}
	\end{figure}
	
	\paragraph{Sine function\\}
	Laplace image: $F(p) = \frac{1}{1 + p^2}.$ Exact Laplace original: $f(t) = \sin t.$ \\
	
	The results of numerical inversion for this case shown in Figure \ref{fig:stehfestsine} behave differently as compared with the previous case. Whereas the deviation from the exact function in the beginning of the curve ($0 \leq t \lesssim 1.2 $) are reasonably accurate, accuracy drastically falls as the value of the argument grows. Even with the values of order as high as 22, the deviation from the exact solution becomes too high already after the first full oscillation (maximum absolute deviation 0.0339 with $N=22$ at $t = 5.01.$).		
	
	\begin{figure}[H]
		\centering
		\includegraphics[width=0.85\linewidth]{contents/part1/figures/stehfest_sine}
		\caption{Results of numerical Laplace inversion by Stehfest's method for sine function.}
		\label{fig:stehfestsine}
	\end{figure}
	
	\paragraph{Damped sine function\\}
	Laplace image: $F(p) = \frac{1}{1 + (p+0.2)^2}.$ Exact Laplace original: $f(t) = \E^{-0.2 t}\sin t.$ \\
	
	\begin{figure}[H]
		\centering
		\includegraphics[width=0.85\linewidth]{contents/part1/figures/stehfest_damped_sine}
		\caption{Results of numerical Laplace inversion by Stehfest's method for damped sine function.}
		\label{fig:stehfestdampedsine}
	\end{figure}
	
	The results of numerical inversion for the damped sine curve are shown in Figure \ref{fig:stehfestdampedsine}. Adding a little damping to the process improves the convergence, to an extent, as can bee seen from the behavior of the curves. Nevertheless, the deviations from the exact curve are still significant and tend to become greater as long as the value of time grows  (maximum absolute deviation 0.00311 with $N=22$ at $t = 4.71.$).		
	
	\paragraph{Linear growth\\}
	Laplace image: $F(p) = \frac{1}{p^2}.$ Exact Laplace original: $f(t) = t.$ \\
	
	\begin{figure}[H]
		\centering
		\includegraphics[width=0.85\linewidth]{contents/part1/figures/stehfest_linear_growth}
		\caption{Results of numerical Laplace inversion by Stehfest's method for linear function.}
		\label{fig:stehfestlineargrowth}
	\end{figure}
	
	The results of numerical inversion are shown in Figure \ref{fig:stehfestlineargrowth}. For the case of linear function, even with a relatively lower value of N=6 the results are already very accurate (maximum absolute deviation 0.0107 with $N=6$ at $t = 5.01.$, $4.84 \cdot 10^{-6}$ with $N = 12$).		
	
	
	\paragraph{Asymmetric distribution\\}
	Laplace image: $F(p) = \frac{1}{(p + 1)^2}.$ Exact Laplace original: $f(t) = t \E^{-t}.$ \\
	\begin{figure}[H]
		\centering
		\includegraphics[width=0.85\linewidth]{contents/part1/figures/stehfest_asymmetric_distribution}
		\caption{Results of numerical Laplace inversion by Stehfest's method for asymmetric distribution.}
		\label{fig:stehfestasymmetricdistribution}
	\end{figure}
	
	The results of numerical inversion are shown in Figure \ref{fig:stehfestasymmetricdistribution}. As can bee seen, the results become reasonably accurate with the values of $N \geq 10$ (maximum absolute deviation 0.00191 with $N=10$ at $t = 4.36.$).		
	
	
	\paragraph{Delayed unitary step\\}
	Laplace image: $F(p) = \frac{\E^{-p}}{p}.$ Exact Laplace original: $f(t) = H(t-1).$ \\
	
	\begin{figure}[H]
		\centering
		\includegraphics[width=0.85\linewidth]{contents/part1/figures/stehfest_delayed_step}
		\caption{Results of numerical Laplace inversion by Stehfest's method for delayed unitary step.}
		\label{fig:stehfestdelayedstep}
	\end{figure}
	
	The results of numerical inversion are shown in Figure \ref{fig:stehfestdelayedstep}. Although the accuracy of the computed values of $f(t)$ with big values of time $t \geq 2.5 $can be considered as more or less accurate, behavior of the calculate curves in the vicinity of the original's breakpoint at $t = 1.$ This behavior cannot be improved by increasing of the order of integration. As can be seen from Table \ref{tab:Stehfest coefficients} the values if coefficients rapidly grow to very big values, as the order $N$ grows, which causes numerical instability due to rounding errors in the machine representation of float numbers.
	
	\paragraph{Rectangular wave\\}
	Laplace image: $F(p) = \frac{(\E^{-p} - \E^{-2p})}{p}.$ Exact Laplace original: $f(t) = H(t-1) - H(t-2).$ \\
	
	
	\begin{figure}[H]
		\centering
		\includegraphics[width=0.85\linewidth]{contents/part1/figures/stehfest_rectanhular_wave}
		\caption{Results of numerical Laplace inversion by Stehfest's method for rectangular wave}
		\label{fig:stehfestrectanhularwave}
	\end{figure}
	
	Like in the case of delayed step, the behavior of the Stehfest approximated originals is far from satisfactory. Whereas in the vicinity of $t=0$ and with gig values of time the accuracy can be considered acceptable, the form of rectangular wave cannot be computed by Stehfest integration method. In this particular case, even the value of integration order as high as 26 did not do better. Further increase in the value of $N$ creates numerical instability, as is shown in Figure \ref{fig:stehfestrectanhularwaveunstable}.
	
	\begin{figure}[H]
		\centering
		\includegraphics[width=0.85\linewidth]{contents/part1/figures/stehfest_rectanhular_wave_unstable}
		\caption{Numerical instability of Stehfest's method for rectangular wave with too high values of integration order.}
		\label{fig:stehfestrectanhularwaveunstable}
	\end{figure}
	
	Setting $N=28$ for the previous example created numerical instability illustrated by the chaotic cloud of black points in the right part of the graph in Figure \ref{fig:stehfestrectanhularwaveunstable}.
	
	\paragraph{Well function\\}
	Laplace image: $F(p) = \frac{1}{p} K_0\left( 2 \sqrt{0.8 p} \right).$ Exact Laplace original: $f(t) = 0.5 E_1\left( \frac{0.8}{t} \right).$ \\
	
	\begin{figure}[H]
		\centering
		\includegraphics[width=0.85\linewidth]{contents/part1/figures/stehfest_well_function}
		\caption{Results of numerical Laplace inversion by Stehfest's method for the well function}
		\label{fig:stehfestwellfunction}
	\end{figure}
	
	The results of numerical inversion are shown in Figure \ref{fig:stehfestwellfunction}. As can bee seen, the results become quite accurate already with the values of $N \geq 6$ (maximum absolute deviation 0.0021 with $N=6$ at $t = 1.01.$).		
	
	\subsection {den Iseger}