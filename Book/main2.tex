\documentclass[12pt,oneside]{book}
% Alternative outline
% Basic packages
\usepackage{fontspec}        % XeLaTeX only
\usepackage{polyglossia}     % multilingual
\setmainlanguage{english}
\usepackage{amsmath}
\usepackage{amsfonts}
\usepackage{amssymb}
\usepackage{xspace}
\usepackage[toc,page]{appendix}
\usepackage[colorlinks=false, pdfborder={0 0 0}]{hyperref}
\usepackage{enumitem}
\usepackage{makeidx}
\usepackage[toc,page]{appendix}

% Table packages
\usepackage{longtable}       % multipage tables
\usepackage{tabularx}        % adjustable-width columns
\usepackage{ltablex}         % cross of longtable+tabularx
\keepXColumns

% Graphics
\usepackage{graphicx}

% Floats
\usepackage{float}

% Misc
\usepackage{lipsum}          % dummy text
\usepackage{booktabs}        % nice tables
\usepackage{dirtree}
\usepackage{url}
\usepackage{color}
\usepackage{hyperref}

% Optional: for subfile editing
\usepackage{subfiles}        % allows chapters to compile independently
%============= Macros ===============
\newcommand{\gw}{groundwater\xspace}
\newcommand{\Gw}{Groundwater\xspace}
\newcommand{\dd}{drawdown\xspace}
\newcommand{\Dd}{Drawdown\xspace}
\newcommand{\gf}{geofiltration\xspace}
\newcommand{\Gf}{Geofiltration\xspace}
\newcommand{\cd}{conductifity\xspace}
\newcommand{\Cd}{Conductifity\xspace}
\newcommand{\pz}{piezoconductifity\xspace}
\newcommand{\Pz}{Piezoconductifity\xspace}

\newcommand{\rc}{recharge\xspace}
\newcommand{\ip}{impermeable\xspace}
\newcommand{\bd}{boundary\xspace}
\newcommand{\fl}{flow\xspace}

\newcommand{\ds}{dynamical system\xspace}
\newcommand{\dss}{dynamical systems\xspace}
\newcommand{\Dss}{Dynamical systems\xspace}
\newcommand{\lds}{linear dynamical system\xspace}
\newcommand{\ldss}{linear dynamical systems\xspace}

\newcommand{\lt}{Laplace transform\xspace}
\newcommand{\lti}{Laplace transform inversion\xspace}
\newcommand{\lop}[1]{\mathcal{L}\{#1\}\xspace}			% Laplace transform operator		
\newcommand{\ilop}[1]{\mathcal{L}^{-1}\{#1\}\xspace}	% Inverted Laplace transform operator

\newcommand{\ode}{ordinary differential equation\xspace}
\newcommand{\odes}{ordinary differential equations\xspace}
\newcommand{\pde}{partial differential equation\xspace}
\newcommand{\pdes}{partial differential equations\xspace}

\newcommand{\tmp}{temporal\xspace}
\newcommand{\tf}{transfer function\xspace}
\newcommand{\tfs}{transfer functions\xspace}
\newcommand{\Tf}{Transfer function\xspace}



\begin{document}

\frontmatter
%\input{frontmatter/title}
%\chapter{Preface}
Motivation \\

\lipsum[1]
%\input{frontmatter/notation}

\tableofcontents

\chapter*{Notation}
\addcontentsline{toc}{chapter}{Notation}
\begin{description}[labelwidth=4em,leftmargin=!]
	\item[$M(x,y)$] Point on Cartesian plane
	\item[$t$] time variable
\end{description}

\mainmatter

\part{Foundations}
\chapter{Preface}
% Here I explain why I put the apparatus before the definition of the sudy, the linear dynamical systems.

\chapter{Mathematical apparatus}
\chapter{ODE}
\chapter{PDE}
\chapter{Laplace Transform}
\chapter{Laplace Transform Inversion}
\chapter{Fourier Transform}
\chapter{Physical phenomena and their equations}
% Electrical processes, heat transfer, diffusion, Darcy, Saint-Venan, ...
	
%\chapter{Cats!}
\lipsum[1-2]
%% !TeX root = ../../main.tex
% !TeX spellcheck = en_US
% !TeX encoding = UTF-8

\chapter{Mathematical foundations}
	\section{Kinds of formalism for the description of linear dynamical systems}
		%	Section 2.1	Kinds of formalism for the description of linear dynamical systems. Mathematical models.
	
	\section{Lumped systems}
		%	Section 2.2	Lumped systems. ODE-based models. Initial conditions. Examples (mechanics, electricity, control systems).
		
	\section{Distributed systems}
		%	Section 2.3	Distributed systems. PDE-based models. Initial and boundary conditions.
		
	\section{Distributed systems described by parabolic PDE}
		%	Subsection 2.3.1	Systems described by parabolic PDE. Examples (heat transfer, diffusion, geofiltration).
		
	\section{Distributed systems described by hyperbolic PDE}
		%	Subsection 2.3.2	Systems described by hyperbolic PDE. Examples (water waves, telegraph signal).
%% !TeX root = ../../main.tex
% !TeX spellcheck = en_US
% !TeX encoding = UTF-8

\chapter{From differential equations to temporal characteristics}
	\section{Direct solution of linear ODEs}
	%	Section 3.1	Direct solution of linear ODEs. Eigenvalues and eigenvectors. Examples.
	
	\section{Laplace Transform.}
		%	Section 3.2	Laplace Transform.
		%	Subsection 3.2.1	Definitions.
		%	Subsection 3.2.2	Properties.
		%	Subsection 3.2.3	Application to the solution of linear ODEs. Examples.
		%	Subsection 3.2.4	Concept of transfer function. Examples (electricity, control systems).
	\section{From Laplace images to temporal characteristics}
		%	Section 3.3	From Laplace images to temporal characterictics.
		%	Subsection 3.3.1	Inversion of LT. Bromwich integral.
		%	Subsection 3.3.2	Inversion tables. Solutions for lumped systems.
		
	\section{Laplace Transform and distributed dynamical systems}
		%	Section 3.4	Laplace Transform and distributed dynamical systems.
		%	Subsection 3.4.1	One-dimensional case. Examples (heat transfer)
		%	Subsection 3.4.2	Two-dimensional case. Problem formulation. Examples (heat transfer)
		%	Subsection 3.4.3	Two-dimensional case. Continuation.
		%	Subsubsection 3.4.3.1	Fourier transform along one spatial dimension. Solution in Fourier space.
		%	Subsubsection 3.4.3.2	Inversion of Fourier transform. Laplace image of the solution. 
		
	\section{Numerical inversion of Laplace transform}
		%	Section 3.5	Numerical inversion of Laplace transform.
		%	Subsection 3.5.1	Overview: traditional approximations
		%	Subsection 3.5.2	Stehfest's method. Examples. Stability.
		%	Subsection 3.5.3	Den Iseger's method. Examples. Problems with Den Iseger (complex computations).
		%	Subsection 3.5.4	Gokhberg's method (historical). Examples. Stability.

\part{Linear Dynamical Systems}
\chapter{Lumped systems}
\chapter{Electrical circuits}
\chapter{Linear models in hydrology} % Rainfall-flow models etc.
\chapter{Lumped models in hydrogeology} 
\chapter{Models of springs} 

\chapter{Distributed systems}
\chapter{Electromagnetic phenomena} % Telegraph equation
\chapter{Heat transfer}
\chapter{Diffusion}
\chapter{Open channel hydraulics}
\chapter{Geofiltration}

%% !TeX root = ../../main.tex
% !TeX spellcheck = en_US
% !TeX encoding = UTF-8

\chapter{What does a geofiltration system consist of?}
	\section{Aquifers}
		%	Section 4.1	Aquifers. Classification of aquifers
		%	Subsection 4.1.1	Geometry
		%	Subsection 4.1.2	Boundary conditions
		%	Subsection 4.1.3	Filtration parameters distribution
		
	\section{Water intakes}
		%	Section 4.2	Water intakes. Classification of water intakes
		%	Subsection 4.2.1	Spatial distribution: single wells, well rows, galleries, area water intakes.
		%	Subsection 4.2.2	Temporal characteristics: constant, step-wise, linear discharge, periodic discharge, stochastic discharge.
		
	\section{State variables}
		%	Section 4.3	State variables. What values are we interested in?
		%	Subsection 4.3.1	Drawdown
		%	Subsection 4.3.1	Groundwater flow (depletion)
		
	\section{Computation of dynamics in linear geofitration systems}
		%	Section 4.4	Computation of dynamics in linear geofitration systems.
		%	Subsection 4.4.1	Transfer function of drawdown
		%	Subsubsection 4.4.1.1	Case study. Transfer function of drawdown in the simplest case (single well in infinite aquifer)
		%	Subsection 4.4.2	Transfer function of flow depletion
		%	Subsubsection 4.4.2.1	Case study. Transfer function of flow depletion in the simplest case (single well in infinite aquifer)
		%	Subsection 4.4.3	Frequency characteristics of aquifer-waterintake systems: what arey for?
		%	Subsubsection 4.4.3.1	Case study. Frequency characteristics of drawdown in the simplest case.
		%	Subsection 4.4.4	Temporal dynamics of drawdown and flow depletion. From Laplace images to time dependencies.
		%	Subsubsection 4.4.4.1	Case study. Temporal dynamics of drawdown in the simplest case. Theis and Hantush solutions.

\part{Linear dynamical models of geofiltration}
\chapter{Semi-infinite aquifers}
\chapter{Quadrant aquifers}
\chapter{Band aquifers}
\chapter{Half-band aquifers}
\chapter{Rectangular aquifers}
\chapter{Wedge aquifers}
\chapter{Circular aquifers}
% % !TeX root = ../../main.tex
% !TeX spellcheck = en_US
% !TeX encoding = UTF-8

\chapter{Model Zoo}

\begin{longtable}{|c|c|c|c|}
	\hline
	Scheme	&  Number code	& Letter code & Description \\
	\hline
	\includegraphics[width=0.15\linewidth]{figures/part_3/I} &  &  &  \\
	\hline
	\includegraphics[width=0.15\linewidth]{figures/part_3/Sr} &  &  &  \\
	\hline
	\includegraphics[width=0.15\linewidth]{figures/part_3/Si} &  &  &  \\
	\hline
	\includegraphics[width=0.15\linewidth]{figures/part_3/Qrr} &  &  &  \\
	\hline
	\includegraphics[width=0.15\linewidth]{figures/part_3/Qir} &  &  &  \\
	\hline
	
	\includegraphics[width=0.15\linewidth]{figures/part_3/Qri} &  &  &  \\
	\hline
	\includegraphics[width=0.15\linewidth]{figures/part_3/Qii} &  &  &  \\
	\hline
	\includegraphics[width=0.15\linewidth]{figures/part_3/Brr} &  &  &  \\
	\hline
	\includegraphics[width=0.15\linewidth]{figures/part_3/Bri} &  &  &  \\
	\hline
	\includegraphics[width=0.15\linewidth]{figures/part_3/Bii} &  &  &  \\
	\hline
	
	\includegraphics[width=0.15\linewidth]{figures/part_3/Hrrr} &  &  &  \\
	\hline
	\includegraphics[width=0.15\linewidth]{figures/part_3/Hirr} &  &  &  \\
	\hline
	\includegraphics[width=0.15\linewidth]{figures/part_3/Hiir} &  &  &  \\
	\hline
	\includegraphics[width=0.15\linewidth]{figures/part_3/Hiii} &  &  &  \\
	\hline
	
	\includegraphics[width=0.15\linewidth]{figures/part_3/Riiii} &  &  &  \\
	\hline
	\includegraphics[width=0.15\linewidth]{figures/part_3/Riiir} &  &  &  \\
	\hline
	\includegraphics[width=0.15\linewidth]{figures/part_3/Riirr} &  &  &  \\
	\hline
	\includegraphics[width=0.15\linewidth]{figures/part_3/Riiir} &  &  &  \\
	\hline
	\includegraphics[width=0.15\linewidth]{figures/part_3/Rriri} &  &  &  \\
	\hline
	\includegraphics[width=0.15\linewidth]{figures/part_3/Rrrrr} &  &  &  \\
	\hline
	
\end{longtable}

\appendix
\input{chapters/appendix/a_01}

\bibliographystyle{ieeetr}
\bibliography{bibliography/references}

\backmatter
\printindex

\end{document}