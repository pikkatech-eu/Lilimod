% !TeX root = ../../main.tex
% !TeX spellcheck = en_US
% !TeX encoding = UTF-8

\chapter{Definitions}

	\section{Dynamical systems}
	% Dynamical systems, their properties and characteristics. Observable values and processes. Examples.
	This book is conceived primarily to concentrate on processes of \gf\footnote{This term seemingly coined by Luckner and Schestakow \cite{luckner_simulation_1975} is, in author's opinion, best to describe the complex of phenomena and models connected with \gw dynamics}. However, the formalism used for the description of that class of phenomena is similar to that used in mathematical models pertaining to phenomena of different nature, decision was made to try to investigate problems arising from a more or less generalized point of view. This approach promises to create a more gentle introduction to the actual models in \gf proper.
	
	We start therefore with a few general definitions. Avoiding to go tofar, however, and trying not to touch deeper domains of pure philosophy, we will not try to formulate a definition for the basic concepts, such as \textbf{system}\index{system}, or \textbf{dynamical system}\index{dynamical system}, and will prefer a pragmatic approach instead. We refer the interested reader to classical works like \cite{Zadeh1963} and \cite{bossel_modeling_2018} that contain attempts to define them at the very fundamental level.
	
	\section{Linear dynamical systems}
	% Section 1.2	Linear dynamical systems. Principle of superposition. Examples.
	
	\section{Classification of linear dynamical systems}
	% Section 1.3	Classification of linear dynamical systems. Lumped and distributed systems. Examples.

% Example how to use \index
% Fourier transform\index{Fourier transform} \lipsum[1]


