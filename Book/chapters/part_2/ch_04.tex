% !TeX root = ../../main.tex
% !TeX spellcheck = en_US
% !TeX encoding = UTF-8

\chapter{What does a geofiltration system consist of?}
	\section{Aquifers}
		%	Section 4.1	Aquifers. Classification of aquifers
		%	Subsection 4.1.1	Geometry
		%	Subsection 4.1.2	Boundary conditions
		%	Subsection 4.1.3	Filtration parameters distribution
		
	\section{Water intakes}
		%	Section 4.2	Water intakes. Classification of water intakes
		%	Subsection 4.2.1	Spatial distribution: single wells, well rows, galleries, area water intakes.
		%	Subsection 4.2.2	Temporal characteristics: constant, step-wise, linear discharge, periodic discharge, stochastic discharge.
		
	\section{State variables}
		%	Section 4.3	State variables. What values are we interested in?
		%	Subsection 4.3.1	Drawdown
		%	Subsection 4.3.1	Groundwater flow (depletion)
		
	\section{Computation of dynamics in linear geofitration systems}
		%	Section 4.4	Computation of dynamics in linear geofitration systems.
		%	Subsection 4.4.1	Transfer function of drawdown
		%	Subsubsection 4.4.1.1	Case study. Transfer function of drawdown in the simplest case (single well in infinite aquifer)
		%	Subsection 4.4.2	Transfer function of flow depletion
		%	Subsubsection 4.4.2.1	Case study. Transfer function of flow depletion in the simplest case (single well in infinite aquifer)
		%	Subsection 4.4.3	Frequency characteristics of aquifer-waterintake systems: what arey for?
		%	Subsubsection 4.4.3.1	Case study. Frequency characteristics of drawdown in the simplest case.
		%	Subsection 4.4.4	Temporal dynamics of drawdown and flow depletion. From Laplace images to time dependencies.
		%	Subsubsection 4.4.4.1	Case study. Temporal dynamics of drawdown in the simplest case. Theis and Hantush solutions.